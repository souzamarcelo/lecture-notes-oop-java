\chapter{Herança e polimorfismo}

\section{Herança}

A herança permite definir elementos específicos, que incorporam a estrutura (atributos) e o comportamento (operações) de elementos mais gerais. Neste sentido, a classe específica herda a estrutura e o comportamento da classe geral, definindo uma hierarquia entre elas. Por conta disso, a herança é também chamada de especialização ou generalização. Com o uso da herança é possível reduzir a escrita de código, diminuindo a redundância e agregando flexibilidade e manutenibilidade ao projeto. Na UML, a herança é representada por uma linha sólida contendo um triângulo no lado da classe mais geral, conforme exemplificado na Figura~\ref{fig:exemplo-heranca-uml}.

\begin{figure}[h]
	\centering
	
	\begin{tikzpicture}
	\umlemptyclass{ClasseA}
	\umlemptyclass[y=-3]{ClasseB}	
	\umlinherit{ClasseB}{ClasseA}
	\end{tikzpicture}
	
	\caption{Representação UML de uma herança}
	\label{fig:exemplo-heranca-uml}
\end{figure}

Consideremos o contexto de uma empresa a qual possui funcionários e gerentes. Estas entidades são modeladas pelas classes \code{Funcionario} e \code{Gerente}. Todo o funcionário da empresa possui uma matrícula e um salário, inclusive os gerentes. Porém, cada gerente possui um número de subordinados e uma senha para acesso ao sistema. Os trechos de código a seguir mostram a implementação das classes \code{Funcionario} e \code{Gerente}.
 
\begin{minted}{java}
public class Funcionario {
	private String matricula;
	private double salario;

	//...
}
\end{minted}
 
\begin{minted}{java}
public class Gerente {
	private String matricula;
	private double salario;
	private int subordinados;
	private int senha;
	
	//...
}
\end{minted}
 
Podemos perceber que há redundância de código, pois ambas as classes definem os atributos \code{matricula} e \code{salario} (linhas 2 e 3 nas duas classes). E se forem incluídos outros tipos de funcionários (secretária, diretor, presidente)? Neste caso, o código deverá ser replicado nas novas classes. E se, após criados os funcionários, seus atributos tiverem que ser alterados? Então cada classe deverá ser alterada. Ou seja, a estrutura adotada não é flexível nem manutenível, o que leva a um software de baixa qualidade.

A solução para este problema está na utilização de herança, de modo que uma classe geral define a estrutura e o comportamento básicos de um funcionário, enquanto classes específicas herdam estas características e adicionam o necessário para cada tipo de funcionário. Neste sentido, a classe \code{Funcionario} passa a ser a classe geral, definindo os atributos \code{matricula} e \code{salario} enquanto a classe \code{Gerente} herda estes atributos e adiciona os atributos específicos de um gerente, isto é, \code{subordinados} e \code{senha}. A estrutura das classes utilizando herança é apresentada pela Figura~\ref{fig:heranca-funcionario-gerente}.

\begin{figure}[h]
	\centering
	
	\begin{tikzpicture}
	\umlclass{Funcionario}{
		-- matricula: String \\
		-- salario: double
	}{}
	
	\umlclass[y=-3.5]{Gerente}	{
		-- subordinados: int \\
		-- senha: int
	}{}
	
	\umlinherit{Gerente}{Funcionario}
	\end{tikzpicture}
	
	\caption{Relacionamento de herança entre \texttt{Funcionario} e \texttt{Gerente}}
	\label{fig:heranca-funcionario-gerente}
\end{figure}

Uma herança define um relacionamento do tipo \textbf{É UM}. Neste caso, um gerente \textbf{É UM} funcionário (ou então, um gerente \textbf{É UM TIPO DE} funcionário). Repare que o contrário não é verdadeiro, isto é, um funcionário \textbf{NÃO É UM} gerente (não necessariamente). Quando lemos o relcionamento da classe mais geral para a mais específica, chamamos de especialização: \textit{um gerente é uma especialização de um funcionário}. Quando lemos o relacionamento da classe mais específica para a mais geral, chamamos de generalização: \textit{um funcionário é uma generalização de um gerente}. A classe geral recebe o nome de \textbf{superclasse} ou \textbf{classe-mãe} (neste caso, a classe \code{Funcionario}), enquanto a classe específica recebe o nome de \textbf{subclasse} ou \textbf{classe-filha} (neste caso, a classe \code{Gerente}).

Os trechos de código abaixo mostram a implementação das classes usando herança, conforme proposto na Figura~\ref{fig:heranca-funcionario-gerente}. A implementação de uma herança é feita utilizando a palavra reservada \code{extends}, onde se define que a classe específica estende a classe geral. Neste caso, \code{Gerente} estende um \code{Funcionario}, pois herda suas características e inclui características adicionais.

\begin{minted}{java}
public class Funcionario {
	private String matricula;
	private double salario;
	
	//...
}
\end{minted}
 
\begin{minted}{java}
public class Gerente extends Funcionario {
	private int subordinados;
	private int senha;
	
	//...
}
\end{minted}

O trecho de código a seguir mostra o uso da estrutura de classes criada. Um objeto da classe \code{Gerente} tem acesso a tudo que for público na classe \code{Funcionario}. Porém, um objeto da classe \code{Funcionario} não herda nada da classe \code{Gerente}. Perceba que o objeto \code{g} acessa os métodos acessores dos atributos \code{matricula} e \code{salario} (linhas 3 e 4), os quais são implementados pela classe \code{Funcionario} e herdados no relacionamento de herança.

\textbf{OBS:} Apesar da classe \code{Gerente} herdar os atributos de \code{Funcionario}, ela não pode acessá-los diretamente, pois são privados (o acesso é feito pelo método acessor correspondente). Uma solução para isso seria utilizar outro modificador de acesso: o \code{protected} -- protegido.

\begin{minted}{java}
public static void main(String[] args) {
	Gerente g = new Gerente();
	g.setMatricula("123456");
	g.setSalario(4500);
	g.setSubordinados(10);
	g.setSenha(1234);
}
\end{minted}
 
A estrutura proposta (e apresentada na Figura~\ref{fig:heranca-funcionario-gerente}) pode crescer, de modo a incluir diferentes tipos de funcionários. Neste caso, cada novo tipo de funcionário é implementado por uma nova classe que herda as características da classe \code{Funcionario}. Com isso, caso no futuro o atributo matrícula seja alterado para um número de registro do tipo \code{int}, a alteração afetará somente a classe \code{Funcionario}. Todos os demais tipos de funcionários passam a herdar as novas caracetrísticas, sem necessidade de alteração do código em diferentes pontos do sistema (isso significa flexibilidade e manutenibilidade). A Figura~\ref{fig:heranca-funcionario-varios} mostra este cenário, com diferentes tipos de funcionário se relacionando com a classe \code{Funcionario} por meio de herança.

\begin{figure}[h]
	\centering
	
	\begin{tikzpicture}
	\umlemptyclass{Funcionario}
	\umlemptyclass[x=-4.25,y=-3.5]{Gerente}
	\umlemptyclass[x=-1.5,y=-3.5]{Diretor}
	\umlemptyclass[x=1.5,y=-3.5]{Presidente}
	\umlemptyclass[x=4.75,y=-3.5]{Secretaria}
	
	\umlinherit{Gerente}{Funcionario}
	\umlinherit{Diretor}{Funcionario}
	\umlinherit{Presidente}{Funcionario}
	\umlinherit{Secretaria}{Funcionario}
	\end{tikzpicture}
	
	\caption{Estrutura de classes herdando as características de \texttt{Funcionario}}
	\label{fig:heranca-funcionario-varios}
\end{figure}
 

\section{Sobrescrita de método}

Consideremos a situação onde todos os funcionários da empresa têm direito a uma gratificação de natal, que consiste em 50\% do seu salário. Logo, podemos implementar o método que determina a gratificação na classe \code{Funcionario}, pois todos os funcionários têm direito a ela. Os códigos a seguir mostram a implementação deste método. 
 
\begin{minted}{java}
public class Funcionario {
	private String matricula;
	private double salario;

	public double gratificacao() {
		return this.salario * 0.5;
	}

	//...
}
\end{minted}
 
\begin{minted}{java}
public class Gerente extends Funcionario {
	private int subordinados;
	private int senha;

	//...
}
\end{minted}
 
Repare que o método \code{gratificacao}, sendo público, também é herdado pelas subclasses de \code{Funcionario}. Com isso, a classe \code{Gerente} herda este método e um objeto desta classe é capaz de chamá-lo. Esta situação é apresentada no código a seguir, que imprimirá em tela o valor de \code{2250.0}, que consiste em 50\% do salário do gerente criado.
 
\begin{minted}{java}
public static void main(String[] args) {
	Gerente g = new Gerente();
	g.setMatricula("123456");
	g.setSalario(4500);
	g.setSubordinados(10);
	g.setSenha(1234);
	System.out.println(g.gratificacao()); //Imprimirá 2250.0
}
\end{minted}
 
 
Consideremos agora que a empresa decida conceder aos gerentes da empresa uma gratificação diferenciada, dado o fato de ser um cargo de confiança. Agora, a gratificação dos gerentes passa a ser 75\% do seu salário. Como resolver este problema?

\textbf{Opção 1:} criar um segundo método chamado \code{gratificacaoGerente}.
\begin{itemize}
	\item \textbf{Problema 1:} a classe \code{Gerente} possuirá dois métodos de gratificação, deixando-a confusa e permitindo a chamada do método errado.
	\item \textbf{Problema 2:} caso a gratificação do diretor seja diferente, um terceiro método deve ser criado, e assim sucessivamente.
\end{itemize}

\textbf{Opção 2 (melhor opção):} reescrever o método \code{gratificacao} na classe \code{Gerente}.
\begin{itemize}
	\item Isso é sobrescrita de métodos!
\end{itemize}

Os trechos de código abaixo mostram a implementação da sobrescrita do método \code{gratificacao}. Agora, a classe \code{Funcionario} possui uma implementação do referido método, enquanto a classe \code{Gerente} possui uma implementação diferente. Todo o objeto da classe \code{Funcionario} executará o método \code{gratificacao} da sua classe e todo o objeto da classe \code{Gerente} executará o método \code{gratificacao} da sua classe.

\textbf{OBS: } uma boa prática é utilizar a anotação \code{@Override}, indicando que o método foi sobrescrito da sua superclasse.

\begin{minted}{java}
public class Funcionario {
	private String matricula;
	private double salario;
	
	public double gratificacao() {
		return this.salario * 0.5;
	}
	
	//...
}
\end{minted}
 
\begin{minted}{java}
public class Gerente extends Funcionario {
	private int subordinados;
	private int senha;

	@Override
	public double gratificacao() {
		return this.getSalario() * 0.75;
	}

	//...
}
\end{minted}
 
O trecho de código abaixo mostra o uso da estrutura de classes criada acima, onde um objeto \code{Gerente} e um objeto \code{Funcionario} são criados. Ao chamar o método \code{gratificacao} do gerente, 75\% do salário é impresso em tela. Ao fazer o mesmo com o funcionário, 50\% do salário é impresso em tela. Ou seja, cada objeto executa o método implementado na sua classe.

\textbf{OBS:} caso o objeto não tivesse implementação do método na sua classe, ele executaria a implementação herdada, isto é, a implementação definida na classe \code{Funcionario}.

\begin{minted}{java}
public static void main(String [] args) {
	Gerente g = new Gerente();
	g.setSalario(1000);
	
	Funcionario f = new Funcionario();
	f.setSalario(1000);
	
	System.out.println(g.gratificacao());  //imprimirá 750.0
	System.out.println(f.gratificacao());  //imprimirá 500.0
}
\end{minted}
 
Neste ponto, se quisermos incluir um novo tipo de funcionário, basta criarmos a classe para o novo tipo e herdarmos as características de \code{Funcionario}. Isso faz com que o novo tipo de funcionário herde os atributos e também o método \code{gratificacao}. Caso queiramos que a gratificação para este novo tipo de funcionário seja diferenciada, basta sobrescrevermos o método \code{gratificacao} (novamente, flexibilidade e manutenibilidade).

Consideremos agora um novo cenário: a gratificação concedida aos gerentes deve ser igual àquela concedida aos funcionários, com um acréscimo de R\$500,00. Uma solução para este novo cenário consiste em implementar a gratificação de 50\% do salário, mais o adicional de R\$500,00 (ou seja, copiar a implementação da classe funcionário, somando ainda o valor adicional). O trecho de código a seguir mostra a implementação do referido método.

\begin{minted}{java}
@Override
public double gratificacao() {
	return this.getSalario() * 0.5 + 500;
}
\end{minted}

No entanto, se a gratificação dos funcionários for alterada (para 60\%, por exemplo), o método da classe \code{Gerente} também deverá ser alterado. Ou seja, uma mudança que exige alteração do código em dois pontos. O mesmo pode ocorrer ainda com diferentes tipos de funcionário.

Uma solução mais adequada consiste em chamar o método \code{gratificaca} da superclasse (\code{Funcionario}) e acrescentar os R\$500,00. O acesso à superclasse é feito através da cláusula \code{super}, que devolve a instância da superclasse da herança, permitindo a execução do método implementado nela. Neste caso, permitindo a chamada do método \code{gratificacao} da classe \code{Funcionario}. O trecho de código abaixo mostra este exemplo.

\begin{minted}{java}
@Override
public double gratificacao() {
	return super.gratificacao() + 500;
}
\end{minted}

Com isso, caso a gratificação dos funcionários seja alterada, apenas o código da classe \code{Funcionario} terá de ser alterado. Sempre que a subclasse deve fazer \textbf{``algo a mais''} em relação à implementação da superclasse, esta técnica deve ser aplicada, garantindo (mais uma vez) a flexibilidade e a manutenibilidade.


\section{Polimorfismo}
 
Na herança desenvolvida nas seções anteriores, um gerente \textbf{É UM} funcionário. Se uma emissora de televisão fizer um convite para que um dos funcionários da empresa conceda uma entrevista, um gerente poderá fazê-lo, pois o gerente é um funcionário. Uma variável do tipo \code{Funcionario} armazena uma referência a um \code{Funcionario}. Logo, ela pode armazenar uma referência a um \code{Gerente}, pois este é um \code{Funcionario}. O trecho de código abaixo exemplifica esta possibilidade.

\begin{minted}{java}
Gerente g1 = new Gerente();
Funcionario f1 = g1;
Funcionario f2 = new Gerente();
\end{minted}

\textbf{Polimorfismo} é, portanto, a capacidade de um objeto poder ser referenciado de várias formas. No exemplo anterior, o objeto \code{f1} pode armazenar uma referência a um objeto da classe \code{Funcionario} ou uma referência a um objeto da classe \code{Gerente}. Se tivéssemos mais classes estendendo \code{Funcionario}, ele poderia armazenar uma referência a um objeto de qualquer uma dessas classes (várias formas -- polimorfismo).

Considerando um objeto do tipo \code{Funcionario} armazenando uma referência a um objeto da classe \code{Gerente}. Ao chamar o método \code{gratificacao} (sobrescrito na classe \code{Gerente}), qual das implementações será executada? Considere que \code{Funcionario} implementa uma gratificação de 50\% e \code{Gerente} implementa uma gratificação de 75\%.

\begin{minted}{java}
Funcionario f = new Gerente();
f.setSalario(1000);
System.out.println(f.gratificacao());
\end{minted}

\textbf{Resposta: } a decisão sobre qual método executar é feito em tempo de execução. O Java verifica qual a classe do objeto que está sendo referenciado dentro da variável e executa o respectivo método. Neste caso, executará o método implementado na classe \code{Gerente} (que é a classe da referência armazenada em \code{f}), imprimindo o valor de \code{750.0}.

O polimorfismo é especialmente útil se quisermos definir um método genérico para todos os funcionários, independente do seu tipo (gerente, diretor, secretária, etc.). A classe apresentada no código abaixo tem a responsabilidade de controlar o total de gratificações concedidas. O método \code{registro} recebe um funcionário e computa a gratificação do mesmo, armazenando a soma total no atributo \code{totalGratificacoes}.

\begin{minted}{java}
public class ControleGratificacoes {
	private double totalGratificacoes = 0;

	public void registro(Funcionario f) {
		this.totalGratificacoes += f.gratificacao();
	}

	public double getTotalGratificacoes() {
		return this.totalGratificacoes;
	}
}
\end{minted}

O método \code{registro} recebe uma referência a um \code{Funcionario}, chamando seu método \code{gratificacao}. Logo, ele pode receber referências a qualquer classe que estende \code{Funcionario}, verificando a referência recebida e chamando o método correto. Ou seja, não importa se o objeto recebido seja um gerente ou uma secretária, o método funcionará de qualquer forma. Mais do que isso, se incluído um novo tipo de funcionário, o método \code{registro} não precisa ser alterado, pois é genérico para qualquer tipo de funcionário (desde que, claro, estenda a classe \code{Funcionario}).

\section{Exemplo -- Veículos}
Considere duas entidades: carro e moto. Um carro possui uma marca, um modelo, uma cor, um valor e um número de portas. Uma moto possui uma marca, um modelo, uma cor, um valor e uma quantidade de cilindradas. Como as classes possuem replicação de código, podemos definir uma classe geral e estendê-la nas classes \code{Carro} e \code{Moto}.

Além dos atributos, estas duas entidades possuem em comum um método para cálculo do seu imposto, que corresponde a 2\% do seu valor. Especificamente para carros, é acrescido R\$\,800,00 ao seu imposto.

\subsection{Estrutura das classes}

A Figura~\ref{fig:exemplo-carro-moto} mostra a estrutura das classes envolvidas no exemplo. A classe \code{Veiculo} define as características gerais de um veículo (marca, modelo cor e valor), bem como o método para cálculo do imposto. As classes \code{Carro} e \code{Moto} herdam os atributos e métodos da classe \code{Veiculo}, adicionando suas particularidades (quantidade de portas e cilindradas), assim como sobrescrevendo o método para cálculo do imposto, no caso dos carros.

\begin{figure}[h]
	\centering
	
	\begin{tikzpicture}
	\umlclass{Veiculo}{
		-- marca: String \\
		-- modelo: String \\
		-- cor: String \\
		-- valor: double
	}{
		+ calculaImposto(): double
	}
	
	\umlclass[x=-3,y=-4]{Carro}	{
		-- qtdPortas: int
	}{
		+ calculaImposto(): double
	}
	
	\umlclass[x=3,y=-4]{Moto}	{
		-- cilindradas: int
	}{}
	
	\umlinherit{Carro}{Veiculo}
	\umlinherit{Moto}{Veiculo}
	\end{tikzpicture}
	
	\caption{Estrutura de classes para o exemplo dos veículos}
	\label{fig:exemplo-carro-moto}
\end{figure}

\subsection{Implementação das entidades}

\subsubsection{Classe \texttt{Veiculo}}

O código abaixo mostra a implementação da classe \code{Veiculo}, que desempenha o papel da superclasse no exemplo. Esta classe define os atributos comuns de todos os veículos (linhas 2 a 5) e a implementação do método comum a todos os veículos (\code{valculaImposto}), o qual define o valor do imposto como 2\% do valor do veículo.

\begin{minted}{java}
public class Veiculo {
	private String marca;
	private String modelo;
	private String cor;
	private double valor;
	
	public double calculaImposto() {
		return this.valor * 0.02;
	}
	
	public Veiculo() {}
	
	public Veiculo(String marca, String modelo, String cor, double valor) {
		this.marca = marca;
		this.modelo = modelo;
		this.cor = cor;
		this.valor = valor;
	}
	
	//Métodos acessores
}
\end{minted}

\subsubsection{Classe \texttt{Carro}}

O código abaixo mostra a implementação da classe \code{Carro}, que herda a estrutura e o comportamento da classe \code{Veiculo} (linha 1) e define suas características específicas, como o atributo \code{numPortas}. Além disso, esta classe sobrescreve o método \code{calculaImposto}, adicionando o valor fixo de R\$\,800,00. Neste caso, o método faz a chamada do cálculo do imposto da superclasse (através do \code{super}) e adiciona um comportamento adicional.

Observe que a mesma característica é observada no método construtor, onde o construtor da superclasse é chamado (linhas 10 e 14) com seus respectivos parâmetros, para então atribuir suas características específicas. Esta é uma boa prática, pois garante que modificações no construtor da superclasse não resultem em alterações nas subclasses.

\begin{minted}{java}
public class Carro extends Veiculo {
	private int numPortas;

	@Override
	public double calculaImposto() {
		return super.calculaImposto() + 800;
	}

	public Carro() {
		super();
	}

	public Carro(int numPortas, String marca, String modelo, String cor, double valor) {
		super(marca, modelo, cor, valor);
		this.numPortas = numPortas;
	}

	//Métodos acessores
}
\end{minted}

\subsubsection{Classe \texttt{Moto}}

O código a seguir mostra a implementação da classe \code{Moto}, que também herda as características da classe \code{Veiculo} e adiciona um atributo específico (\code{cilindradas}). Neste caso, o cálculo do imposto para uma moto é igual ao cálculo geral. Por isso, não é necessária a sobrescrita do método \code{calculaImposto} nesta classe.

\begin{minted}{java}
public class Moto extends Veiculo {
	private int cilindradas;

	public Moto() {
		super();
	}

	public Moto(int cilindradas, String marca, String modelo, String cor, double valor) {
		super(marca, modelo, cor, valor);
		this.cilindradas = cilindradas;
	}

	//Métodos acessores
}
\end{minted}

\subsection{Uso das classes}

Deseja-se implementar os seguintes métodos:
\begin{itemize}
	\item Criação de registros de carros e motos e armazenamento em uma lista polimórfica.
	\item Verificação de veículos de uma determinada marca.
	\item Apresentação dos veículos e seus valores de impostos.
	\item Apresentação de todos os carros da lista.
\end{itemize}

Para isso, criaremos uma classe \code{Aplicacao}, responsável por armazenar a lista de veículos (objetos da classe \code{Veiculo} e, portanto, polimórfica) e implementar os métodos desejados. A Figura~\ref{fig:exemplo-carro-moto-aplicacao} mostra a estrutura de classes atualizada e o trecho de código da sequência apresenta a estrutura básica da classe aplicação.

\begin{figure}[h]
	\centering
	
	\begin{tikzpicture}
	\umlclass{Veiculo}{
		-- marca: String \\
		-- modelo: String \\
		-- cor: String \\
		-- valor: double
	}{
		+ calculaImposto(): double
	}
	
	\umlclass[x=-3,y=-4]{Carro}	{
		-- qtdPortas: int
	}{
		+ calculaImposto(): double
	}
	
	\umlclass[x=3,y=-4]{Moto}	{
		-- cilindradas: int
	}{}
	
	\umlemptyclass[x=-6]{Aplicacao}
	
	\umlinherit{Carro}{Veiculo}
	\umlinherit{Moto}{Veiculo}
	\umldep{Aplicacao}{Veiculo}
	\end{tikzpicture}
	
	\caption{Estrutura de classes atualizada para o exemplo dos veículos}
	\label{fig:exemplo-carro-moto-aplicacao}
\end{figure}

\begin{minted}{java}
public class Aplicacao {
	private List<Veiculo> veiculos = new ArrayList<Veiculo>();
	
	private void criaRegistros() {
		// implementação omitida
	}
	
	private void veiculosDaMarca(String marca) {
		// implementação omitida
	}
	
	private void mostraImpostos() {
		// implementação omitida
	}
	
	private void mostraCarros() {
		// implementação omitida
	}
}
\end{minted}

\subsubsection{Criação dos registros}

O código abaixo mostra a implementação do método \code{criaRegistros}. São criados objetos do tipo \code{Carro} e \code{Moto} e adicionados na mesma lista (\code{veiculos}). Isso funciona pois a lista é de objetos da classe \code{Veiculo} (\code{List<Veiculo>}), que permite armazenar referências a qualquer classe que estenda \code{Veiculo}, que é o caso das classes \code{Carro} e \code{Moto}. Por conta disso, dizemos que a lista é polimórfica.

\begin{minted}{java}
private void criaRegistros() {
	Carro c1= new Carro(2, "VW", "Gol", "prata", 25000);
	Carro c2 = new Carro(4, "Fiat", "Uno", "branco", 20000);
	Carro c3= new Carro(4, "Renault", "Clio", "preto", 32000);
	Carro c4= new Carro(2, "Fiat", "147", "amarelo", 8000);

	Moto m1 = new Moto(150, "Honda", "CG", "azul", 7000);
	Moto m2 = new Moto(150, "Yamaha", "YBR", "vermelho", 12000);

	veiculos.add(c1);
	veiculos.add(c2);
	veiculos.add(c3);
	veiculos.add(c4);
	veiculos.add(m1);
	veiculos.add(m2);
}
\end{minted}

\subsubsection{Consulta de veículos de uma marca}

O código abaixo mostra a implementação do método \code{veiculosDaMarca}. Nele, a lista de veículos é percorrida, independente da referência que se encontra a cada iteração (linha 3). Pela herança, é garantido que todos os objetos dessa lista possuam o método \code{getMarca}, que é utilizado para filtrar os registros e apresentar, ao final, a quantidade de veículos da marca recebida como parâmetro.

\begin{minted}{java}
private void veiculosDaMarca(String marca) {
	int qtd = 0;
	for(Veiculo v: veiculos) {
		if(v.getMarca().equals(marca))
		qtd++;
	}
	JOptionPane.showMessageDialog(null, "A marca " + marca + " possui " + qtd + " veículos!");
}
\end{minted}

\begin{minipage}{\textwidth}
	\textbf{\texttt{SAÍDA:}}\\
	\texttt{A marca VW possui 1 veículos!}\\
	\texttt{A marca Fiat possui 2 veículos!}\\
	\texttt{A marca Renault possui 1 veículos!}\\
	\texttt{A marca Honda possui 1 veículos!}\\
	\texttt{A marca Yamaha possui 1 veículos!}
\end{minipage}

\subsubsection{Apresentação dos impostos}

O código abaixo mostra a implementação do método \code{mostraImpostos}. Em tempo de execução, o Java verifica qual a referência armazenada em \code{v} e executa o respectivo método \code{calculaImposto}. Ou seja, é executado o método implementado em \code{Veiculo} ou em \code{Carro}.

\begin{minted}{java}
private void mostraImpostos() {
	String texto = "";
	for(Veiculo v : veiculos) {
		texto += v.getMarca() + " " + v.getModelo() + "(" + v.getValor() + "): " + v.calculaImposto() + "\n";
	}
	JOptionPane.showMessageDialog(null, texto);
}
\end{minted}

\begin{minipage}{\textwidth}
	\textbf{\texttt{SAÍDA:}}\\
	\texttt{VW Gol (25000.0): 1300.0}\\
	\texttt{Fiat Uno (20000.0): 1200.0}\\
	\texttt{Renault Clio (32000.0): 1440.0}\\
	\texttt{Fiat 147 (8000.0): 960.0}\\
	\texttt{Honda CG (7000.0): 140.0}\\
	\texttt{Yamaha YBR (12000.0): 240.0}
\end{minipage}

\subsubsection{Apresentação dos carros}

O código abaixo mostra a implementação do método \code{mostraCarros}. A lista é percorrida e a classe da referência armazenada em cada posição é verificada pelo método \code{instanceOf}. Este comando verifica se o objeto (à esquerda) é do tipo da classe desejada (à direita), retornando verdadeiro ou falso.

\begin{minted}{java}
private void mostraCarros() {
	String texto = "";
	for(Veiculo v: veiculos) {
		if(v instanceof Carro)
		texto += v.getMarca() + " " + v.getModelo() + ", cor " + v.getCor() + "\n";
	}
	JOptionPane.showMessageDialog(null, texto);
}
\end{minted}

\begin{minipage}{\textwidth}
	\textbf{\texttt{SAÍDA:}}\\
	\texttt{VW Gol, cor prata}\\
	\texttt{Fiat Uno, cor branco}\\
	\texttt{Renault Clio, cor preto}\\
	\texttt{Fiat 147, cor amarelo}
\end{minipage}