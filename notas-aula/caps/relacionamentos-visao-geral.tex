\chapter{Visão geral}

\section{Tipos de relacionamentos}

As classes podem se conter relacionamentos entre si, definindo um vínculo entre seus objetos. Considerando as entidades em negrito, alguns exemplos de relacionamentos entre classes incluem:

\begin{itemize}
	\item Um \textbf{cliente} possui um \textbf{endereço}.
	\item Uma \textbf{empresa} é composta por \textbf{funcionários}.
	\item Uma \textbf{moto} é um tipo de \textbf{veículo}.
	\item Um \textbf{restaurante} possui \textbf{pratos}.
	\item Uma \textbf{correspondência} possui um \textbf{remetente} e um \textbf{destinatário}.
\end{itemize}

Existem diferentes tipos de relacionamentos. A lista abaixo resume os principais tipos de relacionamentos, os quais serão detalhados neste capítulo.

\begin{itemize}
	\item \textbf{Associação:} conexão entre classes.
	\item \textbf{Agregação e composição:} especialização de uma associação onde um todo é relacionado com suas partes (relacionamento ``todo-parte'').
	\item \textbf{Dependência:} um objeto depende de alguma forma de outro (relacionamento de utilização).
	\item \textbf{Herança (generalização):} um dos princípios da orientação a objetos, onde uma nova classe pode ser definida a partir de outra já existente (reutilização).
	\item \textbf{Realização:} um contrato que classe segue (obrigação).
\end{itemize}

\section{Representação UML}

A Figura~\ref{fig:relacionamentos-uml} mostra a representação UML dos relacionamentos supracitados. Uma associação é representada por um segmento de reta que une as duas classes. Os relacionamentos de agregação e composição são representados por uma reta com um losango no lado da entidade que representa o todo. A composição se difere da primeira por apresentar o losango preenchido. A dependência é representada por uma linha tracejada com uma ponta de seta aberta no lado da classe independente. A herança é representada por uma linha com uma ponta de seta fechada no lado da superclasse. Finalmente, a realização é representada da mesma forma que uma herança, com a diferença da reta ser tracejada. O restante deste capítulo discute cada relacionamento entre classes e apresenta sua implementação utilizando a linguagem Java.

\begin{figure}[h]
	\centering
	\begin{tikzpicture}
	\umlemptyclass{A}
	\umlemptyclass[x=6]{B}
	\umlassoc[name=associacao] {A}{B}
	
	\umlemptyclass[y=-2]{A}
	\umlemptyclass[x=6,y=-2]{B}
	\umlaggreg[name=agregacao] {A}{B}
	
	\umlemptyclass[y=-4]{A}
	\umlemptyclass[x=6,y=-4]{B}
	\umlcompo[name=composicao] {A}{B}
	
	\umlemptyclass[y=-6]{A}
	\umlemptyclass[x=6,y=-6]{B}
	\umlimport[name=dependencia] {A}{B}
	
	\umlemptyclass[y=-8]{A}
	\umlemptyclass[x=6,y=-8]{B}
	\umlinherit[name=heranca] {B}{A}
	
	\umlemptyclass[y=-10]{A}
	\umlemptyclass[x=6,y=-10]{B}
	\umlreal[name=realizacao] {A}{B}
	
	\node[above] at (associacao-1) {associação};
	\node[above] at (agregacao-1) {agregação};
	\node[above] at (composicao-1) {composição};
	\node[above] at (dependencia-1) {dependência};
	\node[above] at (heranca-1) {herança};
	\node[above] at (realizacao-1) {realização};
	
	\end{tikzpicture}
	\caption{Representação UML dos relacionamentos entre classes}
	\label{fig:relacionamentos-uml}
\end{figure}