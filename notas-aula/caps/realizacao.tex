\chapter{Realização}

\section{Interfaces}

Uma interface define um contrato ao qual uma classe pode assinar. Este contrato estabelece todos os métodos que esta classe deverá implementar e fornecer aos seus clientes. Quando uma classe assina o contrato (implementa a interface) deve implementar todos os métodos definidos nele.

Considerando o contexto de figuras geométricas apresentado pela Figura~\ref{fig:realizacao-figuras}, a classe abstrata \code{Figura} define o atributo \code{cor} e os métodos \code{area} e \code{perimetro}. Se for necessário que cada figura implemente seu próprio método \code{desenhar}, uma boa estratégia consiste em definir o método abstrato \code{desenhar} na classe \code{Figura}, garantindo que cada subclasse forneça sua implementação, aproveitando-se do polimorfismo.

\begin{figure}[h]
	\centering
	
	\begin{tikzpicture}
	\umlemptyclass[type=abstract]{Figura}
	\umlemptyclass[x=-3,y=-3.5]{Retangulo}
	\umlemptyclass[y=-3.5]{Triangulo}
	\umlemptyclass[x=2.75,y=-3.5]{Circulo}
	
	\umlinherit[geometry=|-]{Retangulo}{Figura}
	\umlinherit{Triangulo}{Figura}
	\umlinherit[geometry=|-]{Circulo}{Figura}
	\end{tikzpicture}
	
	\caption{Estrutura de classes para figuras desenháveis}
	\label{fig:realizacao-figuras}
\end{figure}

Porém, o mesmo sistema possui as classes \code{Fonte}, \code{Serif} e \code{Grossa}, que definem as fontes de texto de um elemento gráfico. Estas classes também devem implementar seus métodos \code{desenhar}. Logo, o método de desenho não é exclusivo das figuras, portanto não fazem parte da sua classe.

A Figura~\ref{fig:realizacao-figuras-fontes} apresenta a situação do diagrama de classes incluindo a estrutura de classes das fontes. Diante disso, qual a forma mais adequada de estruturar o sistema para garantir que tanto as figuras quanto as fontes implementem o método \code{desenhar}? Este método deve estar definido em \code{Figura} ou \code{Fonte}? As figuras devem estender \code{Fonte}? As figuras e as fontes devem estender uma nova classe?

\begin{figure}[h]
	\centering
	
	\begin{tikzpicture}
	\umlemptyclass[type=abstract]{Figura}
	\umlemptyclass[x=-3,y=-3.5]{Retangulo}
	\umlemptyclass[y=-3.5]{Triangulo}
	\umlemptyclass[x=2.75,y=-3.5]{Circulo}
	
	\umlinherit[geometry=|-]{Retangulo}{Figura}
	\umlinherit{Triangulo}{Figura}
	\umlinherit[geometry=|-]{Circulo}{Figura}
	
	\umlemptyclass[x=7,type=abstract]{Fonte}
	\umlemptyclass[x=5.5,y=-3.5]{Serif}
	\umlemptyclass[x=8.5,y=-3.5]{Grossa}
	
	\umlinherit[geometry=|-]{Serif}{Fonte}
	\umlinherit[geometry=|-]{Grossa}{Fonte}
	\end{tikzpicture}
	
	\caption{Estrutura de classes para figuras e fontes desenháveis}
	\label{fig:realizacao-figuras-fontes}
\end{figure}

Adotar herança para resolver este problema não é a solução correta. Uma herança deve ser aplicada estritamente quando o relacionamento entre as classes responde a uma relação \textbf{``É UM''}. Neste caso, uma figura \textbf{não é uma} fonte, e uma fonte \textbf{não é uma} figura. Para resolver este problema, o ideal seria uma forma de apenas definir que as figuras e as fontes devem implementar o método \code{desenhar}. Isso é possível com uma \textbf{interface}!

Uma interface serve para isso, definir um contrato onde as classes que a realizam devem implementar os seus métodos. Com isso, podemos definir uma interface \code{Desenhavel} que exige que as classes que a implementem apresentem a implementação do método \code{desenha}. O trecho de código abaixo mostra a definição de uma interface.

\begin{minted}{java}
public interface Desenhavel {
	void desenhar();
}
\end{minted}

Basta então fazer com que as classes \code{Figura} e \code{Fonte} realizem (ou implementem) a interface criada, fazendo com que suas subclasses tenham que implementar o método de desenho. Logo, o uso de interfaces (ou realização) é diretamente aplicável quando queremos definir um comportamento aplicável a distintas classes, sem que estas tenham um vínculo semântico. Os códigos a seguir mostram estas definições.

\begin{minted}{java}
public abstract class Figura implements Desenhavel {
	private String cor;
	public abstract double area();
	public abstract double perimetro();
	public void desenhar() {
		//implementação aqui.
	}
}
\end{minted}

\begin{minted}{java}
public abstract class Fonte implements Desenhavel {
	public void desenhar() {
		//implementação aqui.
	}
}
\end{minted}

Repare que, neste caso, as classes \code{Figura} e \code{Fonte} não são obrigadas a implementar o método \code{desenhar}, pois são abstratas (o que não impede sua implementação). No entanto, neste caso a obrigatoriedade de implementação é passada às suas subclasses concretas. Se a superclasse \code{Fonte} (por exemplo) implementar o método \code{desenhar}, as classes \code{Serif} e \code{Grossa} não precisam fazê-lo, a não ser que queiram sobrescrever o método. Se a classe \code{Fonte} não implementar o método, isso deve ser feito em \code{Serif} e \code{Grossa}.

A versão final do diagrama de classes é apresentada pela Figura~\ref{fig:realizacao-figuras-completo}. A interface é representada usando o estereótipo \texttt{<<interface>>}. Perceba que entidades de diferentes naturezas (\code{Figura} e \code{Fonte}) implementam a interface. Ou seja, o contrato definido por ela serve para classes sem um vínculo semântico.

\begin{figure}[h]
	\centering
	
	\begin{tikzpicture}
	\umlemptyclass[type=abstract]{Figura}
	\umlemptyclass[x=-3,y=-3.5]{Retangulo}
	\umlemptyclass[y=-3.5]{Triangulo}
	\umlemptyclass[x=2.75,y=-3.5]{Circulo}
	
	\umlinherit[geometry=|-]{Retangulo}{Figura}
	\umlinherit{Triangulo}{Figura}
	\umlinherit[geometry=|-]{Circulo}{Figura}
	
	\umlemptyclass[x=7,type=abstract]{Fonte}
	\umlemptyclass[x=5.5,y=-3.5]{Serif}
	\umlemptyclass[x=8.5,y=-3.5]{Grossa}
	
	\umlinherit[geometry=|-]{Serif}{Fonte}
	\umlinherit[geometry=|-]{Grossa}{Fonte}
	
	\umlemptyclass[x=3.5,y=3,type=interface]{Desenhavel}
	\umlreal[geometry=|-]{Figura}{Desenhavel}
	\umlreal[geometry=|-]{Fonte}{Desenhavel}
	\end{tikzpicture}
	
	\caption{Estrutura de classes para figuras e fontes com interface}
	\label{fig:realizacao-figuras-completo}
\end{figure}

\textbf{OBS:} a realização é representada na UML por uma linha tracejada e um triângulo não preenchido do lado da interface.

\section{Mais polimorfismo}

Com as interfaces, podemos usufruir de uma capacidade ainda maior de polimorfismo. Se tivermos uma lista de figuras (objetos de \code{Figura}), por exemplo, podemos nos referir a elas como desenháveis e chamar os métodos definidos na interface (contrato) \code{Desenhavel}. O trecho de código abaixo exemplifica este caso, onde ambos os laços de repetição são válidos.

\begin{minted}{java}
public void run() {
	List<Figura> figuras = criaListaFiguras();
	
	for(Figura f: figuras) {
		f.desenhar();
	}
	
	for(Desenhavel d: figuras) {
		d.desenhar();
	}
}
\end{minted}

Se futuramente novas classes forem incluídas no sistema como subclasses de \code{Figura}, a aplicação continuará funcionando normalmente, uma vez que a herança e a realização garantem a existência do método \code{desenhar}.

Além disso, é possível criar métodos que operem sobre objetos que implementem a interface \code{Desenhavel}, independente da classe à qual o objeto pertença. Ou seja, as interfaces (realização) fornecem um nível ainda maior de flexibilidade. O trecho de código abaixo exemplifica o uso da realização para criação de rotinas polimórficas.

\begin{minted}{java}
public void desenhaComponente(Desenhavel d) {
	d.desenha();
}

public void desenhaComponentes(List<Desenhavel> lista) {
	for(Desenhavel d: lista) {
		d.desenha();
	}
}
\end{minted}

Se futuramente novas classes que implementem a interface \code{Desenhavel} forem incluídas no sistema, a aplicação continuará funcionando normalmente, uma vez que os métodos \code{desenhaComponente} e \code{desenhaComponentes} fazem uso do polimorfismo oriundo da realização.