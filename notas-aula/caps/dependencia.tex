\chapter{Dependência}

Uma classe \code{A} depende de uma classe \code{B} quando, no momento da compilação da classe \code{A}, o código da classe \code{B} também é compilado. Ou seja, para que a classe \code{A} funcione, é preciso existir (e funcionar) a classe \code{B}. Logo, classes que possuem entre si quaisquer relacionamentos (associação, agregação, composição, especialização) possuem uma dependência.

A dependência é dada pela navegabilidade do relacionamento. Por exemplo, em uma associação a classe que possui um objeto da outra, possui uma dependência com a mesma. Caso a classe independente seja apagada, a classe dependente apresentará erro (ou seja, a classe que implementa o vínculo).

Na orientação a objetos, relacionamentos de diferentes naturezas são representados com diferentes tipos (associação, agregação, composição, etc.). Para os casos onde o relacionamento não se encaixa nos tipos predefinidos, o relacionamento é chamado de dependência. Logo, ele pode ser visto como um relacionamento de utilização, onde os objetos não possuem um vínculo semântico, mas um deles (dependente) faz uso dos serviços do outro (independente). Em geral, uma dependência ocorre quando um \textbf{objeto da outra classe é utilizado como parâmetro ou retorno de um método, ou quando é utilizado internamente pela classe}. As seções a seguir abordam os diferentes casos de dependência através de exemplos.

\section{Dependência no argumento de um método}

Considere as entidades \code{Veiculo} e \code{Posto}. Não existe nenhum relacionamento semântico entre estas classes, isto é, elas não são associadas, nem possuem agregação, composição ou herança. No entanto, a classe \code{Veiculo} deve implementar um método chamado \code{abastecer}, no qual recebe combustível (incrementa seu atributo \code{combustivel}) de um posto de gasolina (classe \code{Posto}), que transfere combustível para o referido veículo (decrementa seu atributo \code{estoque}). Perceba que a entidade \code{Veiculo} usa um serviço fornecido pelo objeto da classe \code{Posto}, mas não possui um relacionamento de associação com esta classe. Logo, o relacionamento consiste em uma dependência.

Podemos implementar esta dependência no argumento do método \code{abastecer}, onde o veículo recebe como parâmetro o posto de combustível no qual está abastecendo, além da quantidade desejada de combustível, para então utilizar o serviço de retirada de combustível (implementado pelo método \code{retiraComb} da classe \code{Posto}). Este cenário é descrito pelo diagrama de classes apresentado na Figura~\ref{fig:dependencia-veiculo-posto} e seu código é apresentado na sequência. Observe que a dependencia e representada por uma linha tracejada com ponta de seta para a classe independente.

\begin{figure}[h]
	\centering
	
	\begin{tikzpicture}
	\umlclass{Veiculo}{
		-- modelo: String \\
		-- placa: String \\
		-- combustivel: double
	}{
		+ abastecer(Posto, double): void
	}
	
	\umlclass[x=8]{Posto}{
		-- nome: String \\
		-- valor: double \\
		-- estoque: double
	}{
		+ retiraComb(double): void
	}
	
	\umldep[geometry=-|-]{Veiculo}{Posto}
	\end{tikzpicture}
	
	\caption{Relacionamento de dependência entre \texttt{Veiculo} e \texttt{Posto}}
	\label{fig:dependencia-veiculo-posto}
\end{figure}

Perceba que o método \code{abastecer} utiliza um objeto da classe \code{Posto} na sua implementação, recebendo-o como argumento. Após isso, a entidade \code{Veiculo} utiliza o serviço implementado pelo método \code{retiraComb} da classe \code{Posto}, capturando seu retorno para o incremento do atributo \code{combustivel}. A classe \code{Posto}, por sua vez, fornece o método \code{retiraComb} que, com base na quantidade recebida como argumento, verifica a possibilidade de abstecimento, decrementando o combustível do seu estoque, em caso positivo. Ao final, o resultado da retirada é devolvido (verdadeiro ou falso).

\begin{minted}{java}
public class Veiculo {
	private String modelo;
	private String placa;
	private double combustivel;
	
	public void abastecer(Posto posto, double qtd) {
		if(posto.retiraComb(qtd))
			this.combustivel += qtd;
	}
	
	//demais métodos
}
\end{minted}

\begin{minted}{java}
public class Posto{
	private String nome;
	private double valor;
	private double estoque;

	public boolean retiraComb(double qtd) {
		if(qtd <= estoque) {
			estoque -= qtd;
			return true;
		}
		return false;
	}
	
	//demais métodos
}
\end{minted}

O relacionamento existente entre as classes \code{Veiculo} e \code{Posto} é de utilização, onde a primeira classe utiliza a segunda. Não existe um vínculo entre as entidades, o que caracteriza a dependência. A classe \code{Veiculo} é dependente da classe \code{Posto}. Observe que a exclusão da classe \code{Posto} implica em erro na classe \code{Veiculo} (pois a última \textbf{depende} da primeira).

\section{Dependência no retorno de um método}

Um segundo tipo de dependência ocorre quando o relacionamento de utilização se dá no retorno do método. Um exemplo disso pode ser observado agregando mais uma característica no exemplo anterior. Se quisermos implementar um método que calcule o desempenho do veículo em termos de emissão de poluentes, emissão de ruído e consumo de combustível. Podemos definir uma classe que agregue as diferentes métricas de desempenho. O código abaixo apresenta a implementação da classe \code{Metrica}, que reúne as três medidas supracitadas.

\begin{minted}{java}
public class Metrica {
	private double emissao;
	private double ruido;
	private double consumo;
	
	public double getEmissao() {
		return emissao;
	}
	
	public void setEmissao(double emissao) {
		this.emissao = emissao;
	}
	
	//demais métodos
}
\end{minted}

Diante disso, queremos implementar um método na classe \code{Veiculo} que calcule as medidas de desempenho em função da aceleração. Logo, o método \code{avaliar} recebe como argumento a aceleração praticada e determina os valores de emissão de poluentes, emissão de ruído e consumo de combustível, reunindo-os em um objeto da classe \code{Metrica}, que é devolvido como retorno do método. A Figura~\ref{fig:dependencia-veiculo-posto-metrica} apresenta o diagrama de classes atualizado. Com isso, a classe \code{Veiculo} depende da classe \code{Metrica} pois implementa um método cujo retorno é um objeto desta classe.

\begin{figure}[h]
	\centering
	
	\begin{tikzpicture}
	\umlclass{Veiculo}{
		-- modelo: String \\
		-- placa: String \\
		-- combustivel: double
	}{
		+ abastecer(Posto, double): void \\
		+ avaliar(double): Metrica
	}
	
	\umlclass[x=8]{Posto}{
		-- nome: String \\
		-- valor: double \\
		-- estoque: double
	}{
		+ retiraComb(double): boolean
	}
	
	\umlclass[y=5]{Metrica}{
		-- emissao: double \\
		-- ruido: double \\
		-- consumo: double
	}{
	}
	
	\umldep{Veiculo}{Posto}
	\umldep{Veiculo}{Metrica}
	\end{tikzpicture}
	
	\caption{Relacionamento de dependência entre \texttt{Veiculo}, \texttt{Posto} e \texttt{Metrica}}
	\label{fig:dependencia-veiculo-posto-metrica}
	\end{figure}

O trecho de código abaixo mostra a implementação atualizada da classe \code{Veiculo}, incluindo o método \code{avaliar}. De acordo com o valor de aceleração, este método define diferentes valores para as medidas reunidas na classe \code{Metrica}. Ao final, o método devolve um objeto \code{Metrica} com os valores definidos.

\begin{minted}{java}
public class Veiculo {
	private String modelo;
	private String placa;
	private double combustivel;
	
	public void abastecer(Posto posto, double qtd) {
		if(posto.retiraComb(qtd))
			this.combustivel += qtd;
	}
	
	public Metrica avaliar(double aceleracao) {
		Metrica m = new Metrica();
		if(aceleracao <= 10) {
			m.setConsumo(12);
			m.setRuido(41);
			m.setEmissao(340);
		} else {
			m.setConsumo(6);
			m.setRuido(70);
			m.setEmissao(510);
		}
		return m;
	}
	
	//demais métodos
}
\end{minted}

\section{Dependência por uso interno}

Um terceiro tipo de dependência ocorre quando uma classe utiliza outra como tipo de atributo, mas não caracterizando como um vínculo de associação, agregação ou composição. Por exemplo, uma classe \code{Aplicacao} que mantém uma lista de objetos da classe \code{Aluno}, na qual armazena os registros recuperados de um banco de dados para apresentação em tela. Neste caso, as entidades envolvidas na relação não possuem vínculo semântico, mas a exclusão da classe \code{Aluno} implica em erro na classe \code{Aplicacao}. Logo, ambas as classes possuem uma dependência. A Figura~\ref{fig:dependencia-aplicacao-aluno} apresenta o diagrama de classes para este exemplo, mostrando que a classe \code{Aplicacao} depende da classe \code{Aluno}. Repare que, neste caso, o atributo que define a dependência pode ser representado no diagrama de classes, evidenciando o tipo de dependência existente entre as classes.

\begin{figure}[h]
	\centering
	
	\begin{tikzpicture}
	\umlclass{Aplicacao}{
		-- alunos: List<Aluno>	
	}{}
	
	\umlclass[x=8]{Aluno}{
		-- nome: String \\
		-- matricula: String \\
		-- fase: int
	}{
		+ métodos construtores
		+ métodos acessores
	}
	
	\umldep{Aplicacao}{Aluno}
	\end{tikzpicture}
	
	\caption{Relacionamento de dependência entre \texttt{Aplicacao} e \texttt{Aluno}}
	\label{fig:dependencia-aplicacao-aluno}
\end{figure}

Os trechos de código abaixo mostram a implementação das classes \code{Aplicacao} e \code{Aluno}. O atributo \code{alunos} define o relacionamento de dependência, conforme apresentado pela Figura~\ref{fig:dependencia-aplicacao-aluno}.

\begin{minted}{java}
public class Aplicacao {
	private List<Aluno> alunos;
	
	public Aplicacao() {
		this.alunos = new ArrayList<Aluno>();
	}
	
	//demais métodos
}
\end{minted}

\begin{minted}{java}
public class Aluno {
	private String nome;
	private String matricula;
	private int fase;
	
	public void setNome(String nome) {
		this.nome = nome;
	}
	
	public String getNome() {
		return this.nome;
	}
	
	//demais métodos
}
\end{minted}

Este tipo de dependência também acontece quando uma classe define um objeto da outra classe dentro de um método. Por exemplo, se uma classe \code{A} deseja usar as operações fornecidas por uma classe \code{B} na implementação de um método. Neste caso, o objeto da classe \code{B} é instanciado dentro do referido método, caracterizando uma dependência entre as classes envolvidas.

\begin{figure}[h]
	\centering
	
	\begin{tikzpicture}
	\umlclass{Aplicacao}{
	}{
		+ processo(): void
	}
	
	\umlclass[x=8]{Utils}{
	}{
		+ calculo(double): double
	}
	
	\umldep{Aplicacao}{Utils}
	\end{tikzpicture}
	
	\caption{Relacionamento de dependência entre \texttt{Aplicacao} e \texttt{Utils}}
	\label{fig:dependencia-aplicacao-utils}
\end{figure}

Um exemplo deste tipo de dependência é apresentado pela Figura~\ref{fig:dependencia-aplicacao-utils}. A classe \code{Utils} é uma classe utilitária, que fornece métodos para outras classes utilizarem. A classe \code{Aplicacao} utiliza a classe \code{Utils}, que fornece o método \code{calculo}. Sua utilização é feita dentro do método \code{processo}. A implementação destas classes é apresentada nos trechos de código abaixo.

\begin{minted}{java}
public class Aplicacao {
	
	public void processo() {
		double valor = 40;
		Utils utils = new Utils();
		double resultado = utils.calculo(valor);
		resultado += 10;
		System.out.println(resultado);
	}
	
	//demais métodos
}
\end{minted}

\begin{minted}{java}
public class Utils {
	
	public double calculo(double valor) {
		return (valor + 40) * (valor / 5);
	}
	
	//demais métodos
}
\end{minted}

Repare que o objeto é criado e instanciado dentro do método \code{processo} da classe \code{Aplicacao}. Logo, trata-se de uma dependência por uso interno.

\textbf{OBS:} o termo \textit{dependência por atributo} não é amplamente utilizado, podendo este tipo de dependência ser encontrado com outros nomes na literatura.