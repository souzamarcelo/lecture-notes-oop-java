\chapter{Classes abstratas}

\section{Conceitos e benefícios}

O maior benefício oriundo da generalização/especialização de classes através da herança é o \textbf{polimorfismo}. Ele é um recurso poderoso, capaz de garantir ao sistema flexibilidade. Para estudar o conceito de classes abstratas e sua relação com o polimorfismo, consideremos um contexto de figuras geométricas, as quais compartilham entre si uma cor e possuem métodos para cálculo da sua área e do seu perímetro. A Figura~\ref{fig:abstratas-figuras} apresenta a estrutura de classes onde diferentes figuras geométricas herdam as características da classe \code{Figura}.

\begin{figure}[h]
	\centering
	
	\begin{tikzpicture}
	\umlemptyclass{Figura}
	\umlemptyclass[x=-3,y=-3.5]{Retangulo}
	\umlemptyclass[y=-3.5]{Triangulo}
	\umlemptyclass[x=2.75,y=-3.5]{Circulo}
	
	\umlinherit[geometry=|-]{Retangulo}{Figura}
	\umlinherit{Triangulo}{Figura}
	\umlinherit[geometry=|-]{Circulo}{Figura}
	\end{tikzpicture}
	
	\caption{Estrutura de classes para a modelagem de figuras geométricas}
	\label{fig:abstratas-figuras}
\end{figure}

Os trechos de código a seguir mostram a implementação das classes \code{Figura} e \code{Retangulo}, envolvidas no diagrama apresentado pela Figura~\ref{fig:abstratas-figuras}. A classe \code{Figura} define o atributo \code{cor} e os métodos \code{area} e \code{perimetro}, que serão herdados pelas subclasses. A classe \code{Retangulo} implementa alguns atributos adicionais, além de sobrescrever os métodos \code{area} e \code{perimetro}.

\begin{minted}{java}
public class Figura {
	private String cor;
	
	public double area() {
		return 0;
	}
	
	public double perimetro() {
		return 0;
	}
}
\end{minted}

\begin{minted}{java}
public class Retangulo extends Figura {
	private double lado1, lado2;
	
	public double area() {
		return lado1 * lado2;
	}

	public double perimetro() {
		return (lado1 * 2) + (lado2 * 2);
	}
}
\end{minted}

Neste exemplo, não faz sentido termos uma instância da classe \code{Figura}, pois uma figura é sempre um triângulo, um retângulo, um círculo, um trapézio, etc. A classe \code{Figura} define uma entidade abstrata, que só existe para definir uma estrutura comum a todas as figuras concretas, além de estabelecer os métodos que todas devem apresentar (cálculo da área e do trapézio).

Tecnicamente, usamos a classe \code{Figura} para garantir a estrutura e o comportamento de todas as figuras e para nos fornecer polimorfismo. Não faz sentido termos uma instância de \code{Figura} e calcularmos sua área ou perímetro, por exemplo.

Uma dica para identificar casos onde não faz sentido termos uma instância da superclasse é analisar seus métodos. Quando não é possível definir sua implementação, é um bom indicativo de que a classe não deve ser instanciada.

No exemplo das figuras geométricas, cada classe concreta (\code{Triangulo}, \code{Retangulo}, etc.) possui sua própria forma de calcular a área e o perímetro em função dos seus atributos. Isso não acontece na classe \code{Figura}, que é incapaz de prover uma implementação dos métodos supracitados.

Nestes casos, podemos definir a classe como \textbf{abstrata}, o que implica na impossibilidade de ser instanciada. Com isso, a classe se restringe a definir a estrutura e o comportamento das classes que a estenderem e fornecer polimorfismo.

Por exemplo, se tivermos uma lista de figuras (\code{List<Figura>}) estamos fazendo uso do polimorfismo. Caso a classe \code{Figura} seja abstrata, temos a garantia de que todos os objetos da lista são figuras concretas (triângulos, retângulos, etc.) e, com isso, faz sentido chamarmos os métodos \code{area} e \code{perimetro} para qualquer um deles.

\section{Implementação}

A palavra reservada \code{abstract} define uma classe como abstrata. O trecho de código abaixo mostra a definição da classe \code{Figura} como abstrata. Com isso, não é possível instanciar objetos dessa classe. O comando \code{Figura f = new Figura()} não compilará, enquanto o comando \code{Figura f = new Circulo()} continua funcionando normalmente.

\begin{minted}{java}
public abstract class Figura {
	private String cor;

	public double area() {
		return 0;
	}
	
	public double perimetro() {
		return 0;
	}
}
\end{minted}

No diagrama de classes, as classes abstratas devem ser representadas como tal, para que esta característica fique facilmente visível. Isso pode ser feito mantendo o nome da classe em itálico (ou ainda adicionando o estereótipo \texttt{<<abstract>>} à classe), conforme apresentado na Figura~\ref{fig:abstratas-figuras-abstract}.

\begin{figure}[h]
	\centering
	
	\begin{tikzpicture}
	\umlemptyclass[type=abstract]{Figura}
	\umlemptyclass[x=-3,y=-3.5]{Retangulo}
	\umlemptyclass[y=-3.5]{Triangulo}
	\umlemptyclass[x=2.75,y=-3.5]{Circulo}
	
	\umlinherit[geometry=|-]{Retangulo}{Figura}
	\umlinherit{Triangulo}{Figura}
	\umlinherit[geometry=|-]{Circulo}{Figura}
	\end{tikzpicture}
	
	\caption{Estrutura de classes para a modelagem de figuras geométricas com abstração}
	\label{fig:abstratas-figuras-abstract}
\end{figure}

\section{Métodos abstratos}

Resolvemos o problema de não fazer sentido termos uma instância da classe \code{Figura}. Porém, a implementação padrão dos métodos \code{area} e \code{perimetro} ainda não fazem sentido na classe \code{Figura}, uma vez que trata-se de uma entidade abstrata. O ideal seria não haver implementação do método, uma vez que não faz sentido. Além disso, também seria ideal exigir que as subclasses implementem o método, uma vez que a implementação padrão não faz sentido para qualqer classe concreta.

Isso é possível definindo os métodos supracitados como abstratos. Com isso, apenas a assinatura do método é definida na superclasse, exigindo que as subclasses os implementem. Com isso, ao obtermos um objeto do tipo \code{Figura}, sabemos que ele é uma referência a um dos tipos concretos e que os métodos \code{area} e \code{perimetro} estão disponíveis (e fazem sentido).

Para definir um método abstrato, basta inserir a palavra \code{abstract} na sua assinatura e omitir sua implementação, colocando ponto-e-vírgula após o fechamento de parêntesis. O trecho de código abaixo mostra a implementação da classe \code{Figura} com a definição dos métodos abstratos. Logicamente, a classe abstrata poderia definir algum método com implementação padrão às suas subclasses, caso isso fizesse sentido (ocorre em outros contextos).

\begin{minted}{java}
public abstract class Figura {
	private String cor;
	public abstract double area();
	public abstract double perimetro();
}
\end{minted}

É importante destacar que somente é possível definir um método como abstrato, caso ele pertence a uma classe abstrata. Classes concretas não podem conter métodos abstratos e devem implementar todos os métodos abstratos herdados. Uma classe abstrata pode definir métodos abstratos e métodos concretos simultaneamente.